% Standard LaTeX document template
%  BE SURE TO PROCESS DOCUMENT TWICE IF IT CONTAINS CROSS-REFERENCES!

\documentclass[12pt]{article}
\usepackage[round]{natbib} %allow to set the bibliography style and
% import the bibliography file. See Bibliography management with
% natbib for more information on
% https://www.sharelatex.com/learn/Bibliography_management_with_natbib.
% See the reference sheet for natbib on
% http://merkel.zoneo.net/Latex/natbib.php.
% Several .bst files can be
% downloaded from http://kinglab.eeb.lsa.umich.edu/pub/biblios/bst/

\usepackage{graphicx,epsfig}
\usepackage{amssymb,amsmath,amsfonts,bm,color,supertabular,longtable,multirow}
\usepackage[colorlinks=true,linkcolor=black,citecolor=black,urlcolor=black]{hyperref}

\setlength{\oddsidemargin}{0in} % left margin, odd pages
\setlength{\evensidemargin}{0in} % left margin, even pages
\setlength{\textwidth}{6.5in} % widtth of text on page
\setlength{\topmargin}{-.3in} % add to default 1 in
\setlength{\headsep}{0in}     % add to default 25pt
\setlength{\textheight}{8.7in}  % height of text on page
\setlength{\parskip}{.1in}            % vertical space between paragraphs
\setcounter{tocdepth}{2}

%\setlength{\parindent}{0in}            % amount of indentation of paragraph


%  newcommands -- more newcommands used in the document.
%  not just in the preamble

\newcommand{\Var}{\mbox{Var}}
\newcommand{\Cov}{\mbox{Cov}}
\newcommand{\E}{\mbox{E}}
\newcommand{\ubeta}{\mbox{\boldmath$\beta$}}
% Independence symbol
\newcommand\independent{\protect\mathpalette{\protect\independenT}{\perp}}
\def\independenT#1#2{\mathrel{\rlap{$#1#2$}\mkern2mu{#1#2}}}


\title{STAT 501 Case Study Assignment} 
\author{Quan Zhao\\
School of Mathematics and Statistics\\ Victoria University of Wellington, New Zealand} 
%\date{}  % Add \date{} to make a blank date.


%  main body of document

\begin{document}

% Titlepage
\maketitle

\begin{abstract}
  This course provides students with an opportunity to develop their
  research skills in Mathematics and Statistics, including use of
  library resources, constructing literature reviews, developing
  research questions, writing research proposals, and developing
  skills in oral presentation. The template file gives an introduction
  to LaTeX,
\end{abstract}


% Table of Contents
\tableofcontents


\setlength{\baselineskip}{0.25in} % min space from bottom of one line

                                 % to top of next in a paragraph

                                 % place after \begin{document}



\newpage  % start from a new page
\section{Introduction}

\label{s.intro}

The relationship between students' interest and aptitude in science and the overall development of countries on a global scale is a subject that is both fascinating and deserving of in-depth analysis. Understanding this relationship can offer valuable insights into the educational, economic, and social fabric of nations.

In this comprehensive study, we aim to shed light on this complex issue by conducting a rigorous statistical analysis at the national level. To achieve this, we have integrated multiple data sources, including the Programme for International Student Assessment (PISA) and the Human Development Index (HDI) from United Nations data. PISA provides a robust measure of 15-year-old students' science literacy, while the HDI offers a composite index of life expectancy, education, and per capita income indicators, which are used to rank countries into tiers of human development.

Our primary objective is to offer a high-level overview that elucidates the relationship between students' scientific capabilities and interests and their country's level of development. By doing so, we aim to identify key issues that may be hindering progress in both educational and developmental sectors. Furthermore, our analysis serves as a foundation for proposing actionable solutions that could be implemented to improve educational outcomes and, by extension, the overall well-being of nations.

Through this study, we hope to contribute to the ongoing discourse on education and development, providing policymakers, educators, and stakeholders with valuable data and insights that can guide future initiatives and reforms.

% Context and background information.
% \cite{Liu05}

% \section{Objectives and Scope}

% %Specific objectives of the statistical consultation.
% %Scope of the analysis including what is and is not covered.

% \begin{itemize}
%   \item Identify the top ten countries in overall sciences scores.
%   \item Determine whether wealthier (higher income) countries tend to provide more support.
%   \item Determine whether students from countries with a higher HDI show more interest in science.
%   \item Determine whether the level of support and interest have an effect on overall science scores.
% \end{itemize}

\section{Objectives and Scope}

\label{s.objectives}

This study aims to provide a comprehensive understanding of the relationship between students' interest and aptitude in science and the overall development of their respective countries. To achieve this, we have outlined the following specific objectives:

\begin{itemize}
\item \textbf{Science Literacy Rankings:} Identify the top ten countries with the highest overall science scores based on PISA assessments. This will serve as a benchmark for evaluating the effectiveness of educational systems in fostering scientific literacy.

\item \textbf{Wealth and Educational Support:} Investigate whether countries with higher per capita income levels tend to provide more educational support in science. Support could be measured in terms of educational funding, availability of science programs, and teacher-to-student ratios.

\item \textbf{Human Development and Scientific Interest:} Examine the correlation between a country's Human Development Index (HDI) and the level of interest in science among its students. Interest could be gauged through survey data, extracurricular participation, or other relevant metrics.

\item \textbf{Impact of Support and Interest:} Assess whether the level of educational support and student interest in science have a significant impact on overall science scores. This will involve multivariate statistical analyses to control for other influencing factors.
\end{itemize}

\textbf{Scope of Analysis:}

The scope of this study is limited to:

\begin{itemize}
\item Data from the most recent PISA assessments, focusing on 15-year-old students' performance in science.

\item HDI data from the United Nations, specifically examining the components related to education and income.

\item Countries for which both PISA and HDI data are available, to ensure a comprehensive and comparative analysis.
\end{itemize}

This study will not cover:

\begin{itemize}
\item In-depth analyses of individual educational systems or curricula.

\item Factors such as cultural attitudes or government policies, unless they are directly related to educational support or student interest in science.
\end{itemize}

By adhering to these objectives and scope, we aim to provide a robust analysis that can inform policy decisions and educational reforms aimed at improving both educational outcomes and national development.

\section{Methodology}

Overview of the statistical methods used.
Software and tools used for analysis.

\section{Data Description}

Sources of data.
Description of variables.
Data collection methods.
Data quality and validation procedures.

\section{Preliminary Data Analysis}

Descriptive statistics.
Data visualization.
Identification of outliers or anomalies.

\section{Statistical Models and Techniques Used}

Statistical tests conducted.
Models fitted to the data.
Model validation techniques.

\section{Key Findings}

Results of the statistical analysis.
Interpretation of these results in context.

\section{Recommendations}

Suggested actions or decisions based on the findings.

\section{Limitations and Future Research}

Limitations in data or methodology.
Recommendations for future research.

\section{Conclusions}

Final summary and conclusions drawn from the statistical analysis.

% end of structure

\newpage
%
%\bibliographystyle{harvard}
\bibliographystyle{apalike3}
%\bibliographystyle{abbrv} %this is the same as plainnat but with last name fist 
%\bibliographystyle{unsrtnat}  % Sets the bibliography style
                              % unsrtnat. See the article about
                              % bibliography styles for more
                              % information on
                              % https://www.sharelatex.com/learn/Natbib_bibliography_styles
\bibliography{BIBTEX_GOF} 


\end{document}


